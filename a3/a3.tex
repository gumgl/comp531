\documentclass{article}

\usepackage{mathtools}
\usepackage[margin=1.5in]{geometry}
\usepackage{float}
\usepackage{verbatim}
\usepackage{graphicx}
\usepackage{amsthm}
\usepackage{xfrac}
\usepackage{tikz} % oh ho...
%\usepackage{svg} % I gave up...

\restylefloat{table}

\author{Guillaume Labranche (260585371)}
\title{COMP 531 -- Advanced Theory of Computation\\Assignment \#3}
\date{due on 7 March 2016}

\newcommand{\R}{\mathbb{R}}
\newcommand{\F}{\mathbb{F}}
%\newcommand{-->}{\rightarrow} %does not work :(
\newtheorem{theorem}{Theorem}
\newtheorem{corollary}{Corollary}
\newtheorem{lemma}{Lemma}
\newtheorem{proposition}{Proposition}

\newenvironment{definition}[1][Definition.]{\begin{trivlist}
\item[\hskip \labelsep {\bfseries #1}]}{\end{trivlist}}

\begin{document}

\pagenumbering{gobble}
\maketitle

\begin{enumerate}
\item Let $r=\sqrt{\log n}$. Following the hint, we divide the input as $r \times r$ blocks containing $\sqrt{\log n} \times \sqrt{\log n} = \log n$ bits each. 
Define the addition function $f_i(x_1,\dotsc,x_r)$ that computes the $i$th bit of the sum of $r$ $r$-bit numbers. Because any formula $s$ with $k$ inputs can be put in CNF or DNF, any function of $k$ inputs can be computed by a circuit of depth 2 with unbounded gates and with size $2^k$. So we can create a circuit for $f_i$ of depth 2 and size $2^{r \times r} = 2^{\log n} = n$. We then need one $f_i$ for each output bit of our block sums. When adding $r$ $r$-bit numbers, the output will be at most $r+\log r$ long. Therefore we need $r + \log r < 2r$ circuits computing $f_i$. This all takes constant depth and polynomial size ($2r \cdot n$).

For each row of blocks, we compute the sum of each block and arrange the results into two rows. The first one concatenating all the sums of even-numbered blocks without shifting the position of the least significant bit of any block sum by padding with 0's. The other row for the odd-numbered blocks. What remains is $2\frac{\log n}{r}=2\frac{\log n}{\sqrt{\log n}}=2\sqrt{\log n}=2r$ rows of $n$-bit integers.

We again separate the input into $r \times r$ blocks. Note that we now have only two rows of blocks and $\frac{n}{r}$ blocks per row. Then we once again compute their sum with parallel $AC^0$ circuits and arrange the results as done above. What remains is 4 rows of $n$-bit numbers. Since addition of two $n$-bit numbers is in $AC^0$, we can perform addition on our 4 numbers and obtain the final sum.

\newpage
\item We show that this problem of multiplying two $n$-bit integers can be $AC^0$-reduced from $PARITY$. If this problem was in $AC^0$ then $PARITY$ would also be. But we showed in class that it is not.

For any number $n$, define $n_1$ as the least significant bit of $n$ and $n_i$ as the $i$th bit.

Let $x$ be the $n$-sized input to the $PARITY$ problem and assume integer multiplication is doable in $AC^0$. From $x$ we insert $\log n$ 0's between every $x_i$ to create a new variable $y$. From $y$ we replace every variable corresponding to an $x_i$ with 1 to create a new variable $z$. Now compute $w=y \cdot z$. Note that $|y|=|z|=n+(n-1)\log n$

Consider the naive way of computing $w$:
$$ w=\sum_{i \in \{j : z_j=1\}}{y \cdot 2^{i-1}} $$

In other words, we start with $w=0$ and for each $i$ such that $z_i=1$, we add $y$ shifted left by $i-1$ bits (i.e. padded with $i$ 0's on the right). Consider the ($n-1+(n-1)\log n$)th bit of $w$. This bit is essentially $\sum_i{x_i} + c_{n-1+(n-1)\log n}$ modulus 2 where $c_i$ is the carry at the $i$th bit in the addition. Therefore $w_{n-1+(n-1)\log n}=PARITY(x_1,\dotsc,x_n,c_{n-1+(n-1)\log n})$. But since we have added $\log n$ 0's between each $x_i$, the carries will get propagated to the $\log n$ 0-columns right of $n-1+(n-1)\log n$ but not further, and $c_{n-1+(n-1)\log n}=0$. Therefore $w_{n-1+(n-1)\log n}=PARITY(x_1,\dotsc,x_n)$ and we have computed $PARITY$ using integer multiplication in $AC^0$, which is in contradiction to the result seen in class. Therefore integer multiplication cannot be computed in $AC^0$.

\item No time %because of COMP-527 midterm and MATH-340 assignment both immediately following the due date. 

\newpage
\item We first make a few assumptions about any PARITY gate:
\begin{itemize}
\item It does not contain both $x$ and $\bar x$ since this always has parity 1 and can be replaced by a constant.
\item It does not contain $x$ or $\bar x$ more than once, since any pair of the same variable has parity 0, and so can be removed without affecting the gate.
\end{itemize}

Let $n$ be the number of variables that are available in the circuit. For $i=0,\dotsc,n$, a PARITY gate can either be fed $x$, $\bar x$ or none of those. Therefore we start with a circuit which has at most $2^{3n}$ PARITY gates. As mentioned in the hint by email, a PARITY gate corresponds to a linear equation of our variables modulus 2. We can build a matrix $A$ from the starting circuit, where each row corresponds to a PARITY gate in the circuit and each column corresponds to a variable. $A$ has $2n$ columns and at most $2^{3n}$ rows. Because all PARITY gates feed into one AND gate, this corresponds to solving the equation $Ax=(1,\dotsc,1)$.

Since the rank of the matrix can never be more than the number of columns, in this case the rank of $A$ is $2n$ or less. We can perform Gaussian elimination on the matrix (without changing the solution set) and end up with at most $2n$ non-zero rows at the top. We can get rid of the all-zero rows. We can then transform $A$ back into a circuit using the scheme described above in reverse.

\end{enumerate}

\end{document}
