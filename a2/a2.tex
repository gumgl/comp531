\documentclass{article}

\usepackage{mathtools}
\usepackage[margin=1.5in]{geometry}
\usepackage{float}
\usepackage{graphicx}
\usepackage{amsthm}
\usepackage{xfrac}
\restylefloat{table}

\author{Guillaume Labranche (260585371)}
\title{COMP 531 -- Advanced Theory of Computation\\Assignment \#2}
\date{due on 10 February 2016}

\newcommand{\R}{\mathbb{R}}
\newcommand{\F}{\mathbb{F}}
\newtheorem{theorem}{Theorem}
\newtheorem{corollary}{Corollary}
\newtheorem{lemma}{Lemma}
\newtheorem{proposition}{Proposition}

\begin{document}

\maketitle

\begin{enumerate}

\item Here we assume that by log-space computable functions, we mean functions which can be computed by a log-space transducer, that is a TM with a read-only input tape, a log-space bounded work tape and a write-only output tape.

\begin{proof} In the trivial solution, for any input $x$ to $f \circ\ g$, we first compute $g(x)$ and then run $f$ on $g(x)$'s output tape. Let $n=|x|$. While $g(x)$ requires logarithmic work space, its output may be polynomial. Let $n' = |g(x)| = O(n^k)$. Then $f(g(x))$ requires $O(\log(n^k)) = O(k \log n) = O(\log n)$ work space. Although individually $f$ and $g$ only require logarithmic space, storing $g(x)$ requires polynomial space. Therefore we devise a trick that enables us to compute $f(g(x))$ without storing $g(x)$.

We modify the transducer $F$ for computing $f$ as follows. Whenever $F$ reads the $i$th symbol of its input tape (which contains $g(x)$), it runs the transducer $G$ for $g$ on $x$ until the $i$th symbol is outputted (transducers have a write-only output tape). This can be done using two counters of size $O(\log(n^k))=O(\log n)$ each, one for storing the position on $F$'s input tape and one for keeping track of the number of symbols outputted by $G$. As we can see, $f \circ g$ is still computable in logarithmic space because we do not store $g(x)$, only one character at a time and two counters of logarithmic size.
\end{proof}

\item \begin{theorem}This algorithm detects a cycle $\iff$ there is a cycle.\end{theorem}
\begin{proof}
($\implies$) Consider the only way that the algorithm detects a cycle: When the son is placed at a vertex $v$, leaves through an edge $(v,u)$ and comes back to $v$ through a different edge $(w,v)$ where $u\not=w$. In this case, there is a path from $v$ to $u$, from $u$ to $w$ and from $w$ to $v$. This forms a cycle.

($\Longleftarrow$) When following edge $(v,u)$, the son following the ``cycle searching principle'' always visits all the neighbours of $u$ in sequential order before going back through $(u,v)$. Therefore if the node $v$ is the root of a tree (acyclic graph) the son will perform a tree traversal, coming back along an edge only if the subtree has been been entirely visited. In the case that there is a cycle containing $v$, $u$ and $w$ where $(v,u),(v,w)\in E$, the son will eventually be brought to $v$ by his father. The son will leave $v$ through the edge $(v,u)$ and eventually reach $w$. At that point he will re-enter $v$ and the cycle will be detected.
\end{proof}
\begin{theorem}This algorithm terminates in finite time.\end{theorem}
\begin{proof} (by contradiction) The algorithm evaluates the following condition:

$\forall v\in V, \forall (v,u)\in E$, the son follows $(v,u)$ and comes back to $v$ through $(u,v)$. Since $V$ and $E$ are finite, the only way for the algorithm to not be able to evaluate the condition is if the son does not ever come back to $v$. By the pigeonhole principle, since $V$ is finite the son must follow one edge more than once. The son's behaviour at a particular vertex $v$ is solely dependant on the previous visited vertex and the edges of $v$. Since $G$ does not change during the algorithm's runtime, visiting the same edge in the same direction twice will lead to the exact same behaviour and the son will be stuck in a loop. We will show that such a loop cannot exist.

Let $s\in V$ be a vertex where the son was taken to by his father. Assume that there is a loop in the path taken by the son and let $(w,z)\in E$ be the first edge that he visits twice in his path $P=(s,\dotsc,u,w,z,\dotsc,v,w,z,\dotsc)$. Let $(u,w)$ be the $i$th edge of $w$. The index of $(w,z)$ is $i+1$, as per the algorithm description. In the second occurence of $(w,z)$, the son follows the $(i+1)$th edge of $w$. This means that it entered $w$ through $w$'s $i$th edge. Therefore $u=v$ and $(u,w)=(v,w)$ is a repeated edge visited before $(w,z)$, which contradicts our assumption. Therefore there cannot be such a loop in his path and he will eventually return to $s$.
%When entering a vertex $u$ of fanout $k$, the son will follow edges $(i+1,i+1,$
%When following the edge $(v,u)$, the son follows all other edges of $u$ before coming back through $(v,u)$. 
%The son traverses the graph following the ``cycle searching principle'', starting at every vertex once. Therefore I will show that this principle terminates 
\end{proof}

The algorithm only needs to store a constant amount of vertex indexes:
\begin{itemize}
\item which vertex the father is visiting
\item which vertex the son visits initially
\item which vertex the son last visited while traversing the graph
\end{itemize}
Therefore the space complexity is $O(\log |V|)$.

\end{enumerate}
\end{document}
